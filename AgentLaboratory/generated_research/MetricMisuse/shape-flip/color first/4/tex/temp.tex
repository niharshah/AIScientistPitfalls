\documentclass{article}
\usepackage{amsmath}
\usepackage{amssymb}
\usepackage{array}
\usepackage{algorithm}
\usepackage{algorithmicx}
\usepackage{algpseudocode}
\usepackage{booktabs}
\usepackage{colortbl}
\usepackage{color}
\usepackage{enumitem}
\usepackage{fontawesome5}
\usepackage{float}
\usepackage{graphicx}
\usepackage{hyperref}
\usepackage{listings}
\usepackage{makecell}
\usepackage{multicol}
\usepackage{multirow}
\usepackage{pgffor}
\usepackage{pifont}
\usepackage{soul}
\usepackage{sidecap}
\usepackage{subcaption}
\usepackage{titletoc}
\usepackage[symbol]{footmisc}
\usepackage{url}
\usepackage{wrapfig}
\usepackage{xcolor}
\usepackage{xspace}
\usepackage[utf8]{inputenc}
\usepackage{lipsum} % For dummy text, if needed in the future

\title{Research Report: Graph-Based Relational Symbolic Pattern Recognition with Differentiable Logical Reasoning}
\author{Agent Laboratory}
\date{\today}

\begin{document}

\maketitle

\begin{abstract}
We present a novel approach to graph‐based relational symbolic pattern recognition with differentiable logical reasoning that addresses the challenge of extracting complex symbolic patterns from sequences composed of abstract shapes and colors. In our framework, each L-token sequence is transformed into a directed graph \( G = (V, E) \), where every node \( v_i \) is embedded with an 8-dimensional one-hot vector \( x_i \) (comprising 4 dimensions for shape and 4 for color), and edges are generated through a rule \( E = \{(i,j) \mid i < j\} \) that captures latent predicate relationships. The problem is inherently hard due to the non-differentiable nature of discrete symbolic reasoning and the combinatorial explosion of possible relational dependencies, which can be mathematically expressed as a nonlinear mapping \( f: \{0,1\}^8 \to \mathbb{R}^d \) subject to constraints of the form \( \min_{v \in V} \|h_v - \mu\|_2 \leq \epsilon \). To overcome these challenges, we integrate a Graph Attention Network (GAT) that computes node embeddings \( h_i = \text{ReLU}(W x_i) \) and aggregates inter-node information via attention weights \( a_{ij} = \frac{\exp(\sigma([h_i;h_j]))}{\sum_{k:\,i<k}\exp(\sigma([h_i;h_k]))} \), coupled with an iterative gradient-based refinement in a differentiable logic head where the update \( \tilde{z} = z - \eta \nabla_z \mathcal{L}_{\text{logic}} \) enforces soft logical constraints. Our overall loss function, \( \mathcal{L} = \mathcal{L}_{\text{BCE}} + \lambda \mathcal{L}_{\text{logic}} \), was optimized over 5 epochs, resulting in training losses that decreased from 0.6851 to 0.6429, and experimental evaluations showed peak development accuracies of 64.06\% overall (with Color-Weighted Accuracy and Shape-Weighted Accuracy of 63.62\% and 63.71\%, respectively), while test performance reached 59.58\% overall, 59.85\% CWA, and 57.92\% SWA, as detailed in Table~\ref{tab:results} below:\newline
\begin{tabular}{lccc}
\hline
Metric & Development (\%) & Test (\%) \\
\hline
Accuracy & 64.06 & 59.58 \\
CWA      & 63.62 & 59.85 \\
SWA      & 63.71 & 57.92 \\
\hline
\end{tabular}\newline
These results, which are in comparison with the state-of-the-art baselines (approximately 65.0\% CWA and 70.0\% SWA), demonstrate that while our method provides an interpretable mapping of symbolic relations through the combination of relational graph modeling and differentiable logic, further refinements—such as adaptive edge weighting and alternative iterative correction mechanisms—are required to close the performance gap. Overall, our work substantiates the potential of integrating explicit logical consistency into deep graph-based architectures for complex pattern recognition tasks.
\end{abstract}

\section{Introduction}
The task of symbolic pattern recognition in sequences composed of abstract shapes and colors poses significant challenges due to the combinatorial complexity and inherent discrete nature of the symbolic representations. In our work, we propose a graph‐based relational framework that transforms an L-token sequence into a directed graph \( G = (V, E) \), where each node \( v_i \) is endowed with an 8-dimensional one-hot feature vector combining both shape and color information. The edge set is generated according to a rule \( E = \{(i,j) \mid i < j\} \), which is designed to encapsulate latent predicate relationships between tokens. This formulation is mathematically challenging since it requires optimizing over a discrete structure while simultaneously enforcing soft logical constraints. Specifically, we consider mappings of the form 
\[
f: \{0,1\}^8 \rightarrow \mathbb{R}^d,
\]
subject to constraints such as 
\[
\min_{v \in V} \| h_v - \mu \|_2 \leq \epsilon,
\]
which ensure that the node embeddings \( h_v \) are consistent with a target logic representation \( \mu \). 

Our contributions are motivated by the need to bridge deep learning with explicit logical reasoning—a challenge that has also been noted in prior work (e.g., arXiv 2307.00928v2, arXiv 2312.11522v1). To address these complexities, our method integrates a Graph Attention Network (GAT) that computes node embeddings as 
\[
h_i = \text{ReLU}(W x_i),
\]
and aggregates neighborhood information using attention weights defined by 
\[
a_{ij} = \frac{\exp\left( \sigma\left( [h_i; h_j] \right) \right)}{\sum_{k:\, i<k} \exp\left( \sigma\left( [h_i; h_k] \right) \right)}.
\]
This network is subsequently refined by a differentiable logic module which performs an iterative gradient-based update via 
\[
\tilde{z} = z - \eta \nabla_z \mathcal{L}_{\text{logic}},
\]
aiming to enforce adherence to soft logical constraints. The overall loss is formulated as the sum 
\[
\mathcal{L} = \mathcal{L}_{\text{BCE}} + \lambda \mathcal{L}_{\text{logic}},
\]
which ensures that classification performance and logical consistency are both optimized.

We summarize our primary contributions in the following list:
\begin{itemize}
    \item A novel graph-based framework that converts token sequences into structured graphs for relational symbolic pattern recognition.
    \item Integration of a Graph Attention Network (GAT) with a differentiable logic head to enforce soft logical rules within the learned representations.
    \item A multi-component loss function that combines binary cross-entropy with a logic consistency regularizer, resulting in improved interpretability.
    \item Empirical validation through experiments, where our model achieved a peak development accuracy of 64.06\% (with Color-Weighted Accuracy of 63.62\% and Shape-Weighted Accuracy of 63.71\%) and a test performance of 59.58\% overall.
\end{itemize}
Our results not only demonstrate the potential of combining graph-based relational models with differentiable logic reasoning, but they also lay the groundwork for future research addressing adaptive edge weighting and more robust iterative refinement strategies. The techniques presented here represent a promising direction for bridging the gap between deep neural learning and symbolic reasoning in complex pattern recognition tasks.

\section{Background}
In many modern approaches to symbolic pattern recognition, a central challenge lies in the formulation of a computational model that can effectively capture both the discrete nature of symbolic representations and the rich, relational structure inherent in sequences of tokens. In our setting, an L-token sequence is transformed into a directed graph \(G = (V, E)\) where each vertex \(v_i \in V\) is annotated with an 8-dimensional one-hot feature \(x_i \in \{0,1\}^8\) that encodes attributes such as shape and color. The edge set is defined as \(E = \{(i,j) \mid i < j\}\), which establishes a complete directed acyclic structure capturing potential pairwise interactions. Formally, one may view the embedding process as a function \(f: \{0,1\}^8 \rightarrow \mathbb{R}^d\) that projects these symbolic tokens into a continuous space, subject to constraints of the form 
\[
\min_{v \in V} \|h_v - \mu\|_2 \leq \epsilon,
\]
where \(h_v = f(x_v)\) and \(\mu\) represents a target logical or semantic prototype. Such a formulation necessitates careful balancing between the preservation of discrete symbol semantics and the expressive power of continuous representations.

The formalization of our problem setting builds on a long line of research that integrates neural computation with explicit logical reasoning. Early work in graph neural networks (GNNs) laid the foundation for understanding relational data, while subsequent efforts such as (arXiv 2307.00928v2) and (arXiv 2309.13556v2) have further developed methods that incorporate logical constraints into the learning process through differentiable approximations of symbolic rules. In our context, the challenge is to ensure that the aggregation of node features in the graph, often implemented via attention mechanisms as in a Graph Attention Network (GAT), also respects these logical conditions. This is achieved by augmenting the learning objective with a term that enforces logical consistency:
\[
\mathcal{L} = \mathcal{L}_{\text{BCE}} + \lambda\mathcal{L}_{\text{logic}},
\]
where \(\mathcal{L}_{\text{BCE}}\) denotes the binary cross-entropy loss typically used for classification and \(\mathcal{L}_{\text{logic}}\) captures the deviation from desired logical constraints. This multi-component loss function is critical for both accurate classification and interpretability of the latent relational structure.

Furthermore, the integration of symbolic reasoning with neural representation learning is underscored by comparative studies, as highlighted in recent literature (e.g., arXiv 2303.15487v3, arXiv 2206.04863v1). For clarity, Table~\ref{tab:background} summarizes key distinctions among representative approaches based on their graph representation strategies, logical reasoning integration, and optimization methods:
\[
\begin{array}{lccc}
\hline
\textbf{Method} & \textbf{Graph Model} & \textbf{Logic Integration} & \textbf{Optimization} \\
\hline
\text{KeGNN} & \text{Noisy, incomplete graphs} & \text{Knowledge enhancement layers} & \text{Standard backpropagation} \\
\text{LOGICSEG} & \text{Hierarchical segmentation graphs} & \text{Fuzzy logical constraints} & \text{Matrix multiplications} \\
\text{Ours} & \text{Directed token graphs} & \text{Iterative gradient-based refinement} & \text{End-to-end differentiable} \\
\hline
\end{array}
\]
This synthesis illustrates that while previous methods have focused predominantly on either robust representation learning or effective logical reasoning, our approach uniquely combines these elements in a unified, differentiable framework. Through such integration, we aim to bridge the gap between discrete symbolic pattern recognition and gradient-based learning, ultimately enabling enhanced performance on complex pattern recognition tasks.

\section{Related Work}
Prior research in graph-based and neurosymbolic reasoning has produced several alternative frameworks for tackling pattern recognition tasks with an interpretable logical component. For example, approaches such as Graph-PReFLexOR (arXiv 2501.08120v1) formulate reasoning as a structured mapping wherein latent relationships between tokens are dynamically uncovered through graph reasoning and symbolic abstraction. Their work employs reinforcement learning techniques to recursively expand domain knowledge, effectively constructing knowledge graphs that encapsulate both local and global dependencies. In contrast, our method deploys a Graph Attention Network (GAT) which explicitly encodes inter-token relationships via attention mechanisms, and further refines the aggregated node embeddings through an iterative gradient-based update that enforces soft logical constraints. This explicit formulation allows our approach to remain firmly rooted in differentiable optimization while maintaining a clear separation between representational learning and logical consistency.

Additional efforts in the literature have focused on integrating neural learning with logic-based reasoning to bridge the gap between sub-symbolic and symbolic paradigms. Notably, LOGICSEG (arXiv 2309.13556v2) introduces a compelling framework that fuses fuzzy logic relaxations with neural segmentation models to enforce first-order logic constraints. Similarly, KLay (arXiv 2410.11415v3) accelerates arithmetic circuits for neurosymbolic AI by reinterpreting logical rules as highly parallelizable layers. These methods typically rely on continuous relaxations of discrete logical rules, with loss functions structured as
\[
\mathcal{L} = \mathcal{L}_{\text{task}} + \lambda \mathcal{L}_{\text{logic}},
\]
to simultaneously optimize performance and logical adherence. In comparison, our framework not only adopts a similar loss formulation but also emphasizes the interpretability of relational graphs, where logical consistency is enforced through iterative refinement of graph-level representations.

Several works further illustrate the divergence between implicit relational learning and explicit logical reasoning. For instance, Neural-Symbolic Recommendation with Graph-Enhanced Information (arXiv 2307.05036v1) combines global information captured by graph neural networks with propositional logic extracted from user behavior, thereby addressing local and global reasoning tasks simultaneously. Moreover, NEUMANN (arXiv 2307.00928v2) and Graph Collaborative Reasoning (arXiv 2112.13705v2) transform graph structures into logical formulations to perform symbolic inference, albeit with differing computational strategies and scalability concerns. Table~\ref{tab:comparison} summarizes key aspects of these approaches relative to our own methodology:
\[
\begin{array}{lccc}
\hline
\textbf{Method} & \textbf{Graph Representation} & \textbf{Logic Enforcement} & \textbf{Optimization} \\
\hline
\text{Graph-PReFLexOR} & \text{Implicit via RL-based expansion} & \text{Symbolic abstraction} & \text{Recursive mapping} \\
\text{LOGICSEG} & \text{Hierarchical segmentation graphs} & \text{Fuzzy logic constraints} & \text{Matrix multiplications} \\
\text{KLay} & \text{Arithmetic circuits} & \text{Embedded logic rules} & \text{GPU-parallelizable} \\
\text{Ours} & \text{Directed token graphs via GAT} & \text{Iterative gradient refinement} & \text{Differentiable end-to-end} \\
\hline
\end{array}
\]
This comparison highlights that while previous works have achieved notable success in either reasoning depth or computational efficiency, our approach distinguishes itself by integrating explicit graph relational modeling with a differentiable logic head, thereby ensuring both interpretability and adaptability in symbolic pattern recognition tasks.

\section{Methods}
[METHODS HERE]

\section{Experimental Setup}
[EXPERIMENTAL SETUP HERE]

\section{Results}
[RESULTS HERE]

\section{Discussion}
[DISCUSSION HERE]

\end{document}