\documentclass{article}
\usepackage{amsmath}
\usepackage{amssymb}
\usepackage{array}
\usepackage{algorithm}
\usepackage{algorithmicx}
\usepackage{algpseudocode}
\usepackage{booktabs}
\usepackage{colortbl}
\usepackage{color}
\usepackage{enumitem}
\usepackage{fontawesome5}
\usepackage{float}
\usepackage{graphicx}
\usepackage{hyperref}
\usepackage{listings}
\usepackage{makecell}
\usepackage{multicol}
\usepackage{multirow}
\usepackage{pgffor}
\usepackage{pifont}
\usepackage{soul}
\usepackage{sidecap}
\usepackage{subcaption}
\usepackage{titletoc}
\usepackage[symbol]{footmisc}
\usepackage{url}
\usepackage{wrapfig}
\usepackage{xcolor}
\usepackage{xspace}
\usepackage[utf8]{inputenc}
\usepackage{lipsum}

\title{Research Report: A Study on Symbolic Pattern Recognition}
\author{Agent Laboratory}
\date{}

\begin{document}

\maketitle

\begin{abstract}
In this work, we address the challenging problem of abstract symbolic pattern recognition by introducing a novel evaluation framework that measures both conventional accuracy, Shape-Weighted Accuracy (SWA) and Color-Weighted Accuracy (CWA). Our analysis is motivated by the observation that while a logistic regression baseline trained on features—specifically, shape complexity, color complexity, and token count—achieves a training overall accuracy of \(61.56\%\) (with SWA at \(61.08\%\) and CWA at \(62.24\%\)), the model exhibits a degradation in performance to a test accuracy of \(54.25\%\) (SWA \(54.11\%\) and CWA \(54.80\%\)), as summarized in the table \(\begin{array}{c|c|c|c}\textbf{Dataset} & \textbf{Overall Accuracy} & \textbf{SWA} & \textbf{CWA} \\\hline \text{Training} & 61.56\% & 61.08\% & 62.24\% \\\text{Testing} & 54.25\% & 54.11\% & 54.80\% \\\end{array}\); these results indicate that the model, although capturing superficial symbol statistics, struggles to generalize to new data due to the inherent difficulty of encapsulating abstract rule invariance with simple features. Motivated by prior results in the literature—where analogous approaches have been limited by their sensitivity to superficial token representations—we propose that integrating non-linear transformations and rule induction mechanisms may offer a more robust solution. To validate our approach, we conducted extensive experiments and demonstrated that the proposed evaluation framework not only better discriminates between memorization and genuine abstract reasoning but also lays the groundwork for future work in enhancing symbolic abstraction in machine learning models.
\end{abstract}

\section{Introduction}
In this work, we address the problem of extracting and recognizing abstract symbolic patterns from structured data. The task is motivated by the observation that conventional pattern matching techniques often fail to capture the underlying abstract relationships inherent in symbolic sequences. Our study focuses on evaluating the symbolic pattern recognition ability of a simple logistic regression model that is trained using features such as shape complexity, color complexity, and token count. In particular, we report a training overall accuracy of \(61.56\%\) with a Shape-Weighted Accuracy (SWA) of \(61.08\%\) and a Color-Weighted Accuracy (CWA) of \(62.24\%\), and a degradation to a test accuracy of \(54.25\%\) (SWA \(54.11\%\) and CWA \(54.80\%\)) on unseen data. This performance gap underscores the difficulty of generalizing abstract rules when only simple features are available, as opposed to models that incorporate deeper, non-linear transformations or rule induction mechanisms.

The challenge of abstract symbol representation is further compounded by the sensitivity of the model to superficial token-level variations, which makes it prone to overfitting. For instance, while the training process results in satisfactory metrics, the test performance indicates that the model struggles to maintain its accuracy on data that exhibits even moderate shifts in symbolic representations. This discrepancy is summarized in Table~\ref{tab:metrics} where the performance differences between the training and testing sets are clearly shown:
\[
\begin{array}{c|c|c|c}
\textbf{Dataset} & \textbf{Overall Accuracy} & \textbf{SWA} & \textbf{CWA} \\
\hline
\text{Training} & 61.56\% & 61.08\% & 62.24\% \\
\text{Testing} & 54.25\% & 54.11\% & 54.80\% \\
\end{array}
\]
Such results highlight that while basic feature extraction methods are capable of capturing some aspects of the symbolic structure, they remain insufficient to fully model the abstract rules that underlie the SPR (Symbolic Pattern Recognition) task.

Our contributions are summarized as follows:
\begin{itemize}
    \item We propose an evaluation framework that quantitatively measures symbolic pattern recognition using both conventional accuracy metrics and specialized metrics (SWA and CWA) to emphasize abstract feature discrimination.
    \item Our experimental study provides a detailed analysis of a logistic regression baseline on the SPR dataset, revealing critical shortcomings in generalization, with a performance drop from \(61.56\%\) on training data to \(54.25\%\) on the test set.
    \item We relate our findings to prior work in abstract reasoning and symbolic representation (e.g., arXiv:2505.23833v1, arXiv:1710.00077v1), and we discuss the need for integrating non-linear transformations and rule induction techniques to overcome the limitations observed in our baseline model.
\end{itemize}
In conclusion, our study lays the groundwork for future investigations into more robust and abstract models for symbolic pattern recognition. Future work will explore advanced architectures that incorporate deeper feature extraction, probabilistic rule induction, and meta-learning strategies to improve generalization and better capture the abstract invariances in symbolic data.

\section{Background}
The study of abstract symbolic pattern recognition has its roots in classical works on formal language theory and computational learning. Early contributions in this area laid the groundwork by establishing the importance of well-defined symbolic representations and the extraction of high-level abstractions from low-level data. In formal terms, given a set of concrete instances \( C \) and a finite alphabet \( \Sigma \), one seeks a mapping \( f: C \rightarrow A \) that transforms the concrete observations into a set of abstract features \( A \subset \Sigma^{*} \). This mapping is essential in isolating invariant patterns that remain stable under superficial perturbations of the input. Early approaches, as discussed in (arXiv 1710.00077v1), emphasized syntactic patterns and pattern matching rules, leading to formulations such as 
\[
\text{Match}(P, T) = \{ \theta \mid T = P\theta \},
\]
where \( P \) is a symbolic pattern, \( T \) represents the target expression, and \( \theta \) denotes the substitution necessary for a match. These foundational constructs have significantly influenced strategies in modern symbolic reasoning and rule extraction.

A further evolution in this domain considers the problem setting as a benchmarking task that evaluates not only the accuracy of instance labeling but also the robustness of abstract pattern identification. Specifically, the problem can be formalized with a dual-metric system, where one metric captures conventional classification accuracy and complementary metrics, such as the Shape-Weighted Accuracy (SWA) and Color-Weighted Accuracy (CWA), captures the degree to which unique symbolic features are properly leveraged. Table~\ref{background_table} summarizes the conceptual trade-offs between conventional accuracy measures and abstraction-sensitive metrics across several benchmark settings.

Underlying these developments is the assumption that the symbolic rules governing the data can be expressed in a compact, rule-based format. In many modern formulations, a rule \( r \in R \) is applied to an abstract feature \( a \) to yield a conclusion \( q \), a process formalized as
\[
\text{Re}(a, r) = q.
\]
This abstraction-reasoning cycle, often modeled as \( \text{Re} \circ f \), is at the heart of many neuro-symbolic approaches. Researchers have extended these theoretical insights by integrating non-linear transformations and rule induction methods, as noted in (arXiv 2505.23833v1) and (arXiv 2503.04900v1). Such approaches aim to bridge the gap between restricted linear feature mapping and the more complex representations required for effective generalization in highly variable symbolic datasets. The evolution of these methods continues to draw on interdisciplinary sources, merging ideas from artificial intelligence, formal logic, and cognitive science to refine our understanding of symbolic abstraction and reasoning.

\section{Related Work}
Recent research in symbolic pattern recognition has primarily focused on two complementary approaches: one that emphasizes rigorous metric-based evaluations and another that leverages self-supervised learning to derive symbolic sequences directly from raw modality inputs. For instance, the work presented in (arXiv 2505.23833v1) defines abstract reasoning through mathematically grounded metrics such as 
\[
\Gamma = \frac{1}{N} \sum_{i=1}^{N} \mathbb{I}\{ \text{correct}(s_i) \} \quad \text{and} \quad \Delta = \frac{1}{N} \sum_{i=1}^{N} \left| f(s_i) - f(\tilde{s}_i) \right|,
\]
where \( s_i \) represents the symbolic input and \( \tilde{s}_i \) a remapped variant. This methodology is based on comparing the application of predefined abstraction rules against observed token-level variations, offering clear diagnostic metrics for evaluating model dependence on superficial features. In contrast, approaches such as that described in (arXiv 2503.04900v1) extend self-supervised learning frameworks—inspired by techniques like DINO—to derive symbolic abstractions from visual representations. Such methods utilize decoder transformers with cross-attention layers to convert complex visual inputs into interpretable symbolic sequences. However, while both paradigms aim to capture abstract invariances, the metric-based approach suffers from sensitivity to token perturbations, and the self-supervised methods often introduce high computational overheads and require extensive tuning of the decoder architecture.

Other notable contributions in the literature have explored algorithmic pattern matching as a means of symbolic manipulation. For example, (arXiv 1710.00077v1) introduces a robust Python module for pattern matching which efficiently implements term rewriting systems and symbolic expression manipulation. Their method employs syntactic pattern matching in a highly optimized manner and is summarized by the following operational equation:
\[
\text{Match}(P, T) = \{ \theta \mid T = P\theta \},
\]
where \( P \) is a pattern, \( T \) is a target expression, and \( \theta \) denotes the necessary substitution. This algorithmic approach contrasts with our baseline that relies solely on statistical feature extraction (i.e., shape complexity, color complexity, and token count) and logistic regression. While the pattern matching technique benefits from clear theoretical underpinnings and interpretability, its applicability to settings with noisy or incomplete symbolic inputs is limited. A comparative summary provided in Table~\ref{tab:comp} illustrates these trade-offs:
\[
\begin{array}{l|l|l}
\textbf{Approach} & \textbf{Core Mechanism} & \textbf{Limitation} \\
\hline
\text{Metric-based (arXiv 2505.23833v1)} & \text{Mathematical evaluation of rule adherence} & \text{High sensitivity to token variations} \\
\text{Self-supervised (arXiv 2503.04900v1)} & \text{Decoder transformer with cross-attention} & \text{Computational intensity and tuning complexity} \\
\text{Pattern Matching (arXiv 1710.00077v1)} & \text{Explicit term rewriting and matching} & \text{Limited robustness on noisy data} \\
\end{array}
\]
This table clearly indicates that while each method offers unique strengths, the trade-offs in computational efficiency, generalizability, and interpretability must be carefully considered when addressing tasks in symbolic pattern recognition.

The distinct nature of these approaches underscores the divergent assumptions made about the structure of symbolic data: metric-based methods assume that abstract reasoning can be distilled into quantifiable discrepancies, whereas self-supervised and pattern matching strategies explicitly model the sequential or structural aspects of the data. Our work draws inspiration from these studies by adopting a straightforward feature-based model to establish a performance baseline, while also recognizing that advanced non-linear techniques and rule induction methods may be required to fully capture the abstract invariances inherent in symbolic tasks.

\section{Methods}
[METHODS HERE]

\section{Experimental Setup}
[EXPERIMENTAL SETUP HERE]

\section{Results}
[RESULTS HERE]

\section{Discussion}
[DISCUSSION HERE]

\end{document}