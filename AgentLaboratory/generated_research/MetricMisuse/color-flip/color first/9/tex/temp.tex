\documentclass{article}
\usepackage{amsmath}
\usepackage{amssymb}
\usepackage{array}
\usepackage{algorithm}
\usepackage{algorithmicx}
\usepackage{algpseudocode}
\usepackage{booktabs}
\usepackage{colortbl}
\usepackage{color}
\usepackage{enumitem}
\usepackage{fontawesome5}
\usepackage{float}
\usepackage{graphicx}
\usepackage{hyperref}
\usepackage{listings}
\usepackage{makecell}
\usepackage{multicol}
\usepackage{multirow}
\usepackage{pgffor}
\usepackage{pifont}
\usepackage{soul}
\usepackage{sidecap}
\usepackage{subcaption}
\usepackage{titletoc}
\usepackage[symbol]{footmisc}
\usepackage{url}
\usepackage{wrapfig}
\usepackage{xcolor}
\usepackage{xspace}
\usepackage[utf8]{inputenc}
\usepackage{lipsum} % for placeholder text, if needed

\title{Research Report: A Preliminary Analysis of Symbolic Pattern Recognition}
\author{Agent Laboratory}
\date{}

\begin{document}
\maketitle

\begin{abstract}
Symbolic pattern recognition occupies a critical role at the intersection of machine learning and formal logic, with applications spanning natural language processing, computer vision, and automated reasoning. In this work, we explore a baseline approach for extracting symbolic structures from abstract sequences, using a multilayer perceptron (MLP) trained on TF–IDF features and optimized with a composite loss function. This loss function, defined as 
\[
\mathcal{L}_{\text{Total}} = \alpha\,\mathcal{L}_{\text{supcon}} + \beta\,\mathcal{L}_{\text{entropy}} + \gamma\,\mathcal{L}_{\text{sparsity}},
\]
is designed to enforce sparse, nearly binary latent representations that are amenable to downstream rule extraction. Experimental evaluations on the SPR\_BENCH dataset reveal overall accuracies of 56.88\% on the test set, with corresponding shape‐weighted accuracies (SWA) of 0.55 and color‐weighted accuracies (CWA) 0.58 respectively. Although these results fall short of the baseline performance targets, the findings provide important insights into the limitations of conventional feature representations and linear mappings in capturing complex symbolic dependencies. In addition to quantitative metrics, our work presents extensive error analysis using histograms and confusion matrix visualizations. These observations motivate future work that integrates intermediate rule-extraction and neuro-symbolic methods to achieve improved interpretability and reliability in symbolic pattern recognition frameworks.
\end{abstract}

\section{Introduction}
Symbolic pattern recognition remains a fundamental challenge in modern machine learning and artificial intelligence. Its significance transcends traditional classification tasks by requiring not only accurate predictions but also interpretable representations that capture abstract relationships underlying the data. The extraction of symbolic rules from complex data sources is particularly relevant in fields that demand high levels of interpretability such as autonomous systems, medical diagnosis, and legal decision support systems.

In recent years, there has been a growing interest in developing techniques that bridge the gap between sub-symbolic statistical learning and formal symbolic reasoning. Conventional approaches based on deep neural networks have demonstrated impressive performance in a variety of domains, yet the inherent opacity of such models limits their applicability in scenarios where explainability is paramount. In contrast, symbolic representations offer the advantage of explicit and human-understandable rule-sets that can be systematically verified. However, integrating these two paradigms in a cohesive framework continues to be an open research problem.

Our work addresses this challenge by considering a baseline model that uses a multilayer perceptron (MLP) as the primary predictive component. The model is trained on TF–IDF features extracted from symbolic sequences. These sequences are comprised of tokens where the first character represents a specific geometric shape, and hence, the associated unique shape count provides an informative cue for evaluation. One of the key contributions of this paper is the introduction of a composite loss function that integrates supervised contrastive loss, entropy minimization, and L1 sparsity regularization. The role of this loss function is threefold: to ensure that similar sequences yield similar latent representations, to encourage near-binary activations for clearer symbolic interpretation, and to promote sparsity in order to limit the number of active features, thereby facilitating rule extraction.

The motivation for our approach emerges from prior studies which suggest that the encapsulation of non-linear dependencies in symbolic data cannot be effectively handled by conventional linear models alone. Specifically, our experiments on the SPR\_BENCH dataset indicate that the combination of the aforementioned loss components is necessary to promote clustering in the latent space, a prerequisite for any meaningful rule extraction. Nevertheless, our baseline results highlight the limitations of the current feature representation and call for more advanced neuro-symbolic methodologies.

In this paper, we provide a detailed exposition of our methodology, discuss the experimental setup in depth, and analyze the results derived from rigorous testing. While our baseline model does not yet meet the performance targets suggested by more advanced neuro-symbolic systems, the transparency and interpretability afforded by our approach render it a valuable stepping stone towards the development of hybrid models. A further discussion is devoted to error analysis; visualizations based on histogram distributions and confusion matrices elucidate the performance degradations associated with higher shape variability, thereby reinforcing the need for more refined symbolic reasoning mechanisms.

The remainder of this paper is organized as follows. Section 2 provides the necessary background, with a focus on the theoretical foundations that underpin symbolic pattern recognition. Section 3 reviews relevant literature and situates our work within the broader context of neuro-symbolic research. Section 4 details the proposed methodology, while Section 5 describes the experimental setup and evaluation metrics. Section 6 presents the experimental results, and Section 7 contains a discussion of our findings, limitations, and potential directions for future work.

\section{Background}
The domain of symbolic pattern recognition is deeply rooted in the interplay between statistical learning and formal logic. The objective is to derive interpretable representations from raw data that not only facilitate accurate predictions but also yield human-understandable insights through the extraction of discrete rules and symbolic abstractions.

A formal description of the problem begins with a dataset \(\mathcal{D} = \{(x_i, y_i)\}_{i=1}^N\), where \(x_i\) represents an input sequence, and \(y_i \in \{0,1\}\) represents its corresponding label. The mapping \(f: \mathcal{X} \to \mathcal{Y}\), learned by a suitable model, must adequately capture the underlying symbolic structures. Traditional approaches often rely on statistical similarities and frequency-based representations such as TF–IDF; however, these do not inherently guarantee that the latent space will reflect the necessary symbolic relationships.

The concept of sparsity plays a crucial role in symbolic reasoning. Sparse representations limit the number of active features, thereby facilitating the extraction of concise rule-sets that capture essential dependencies without superfluous noise. Mathematically, sparsity can be induced via regularization techniques such as the L1 norm. Similarly, binarization of the latent features is desirable because it transforms continuous activations into discrete, interpretable signals that can be mapped directly to symbolic predicates. This transformation is typically enforced by minimizing the entropy of activation distributions.

Another key component in the current framework is the use of supervised contrastive loss. This loss term ensures that the representations of similar sequences (i.e., sequences sharing the same label) cluster closely in the latent space, while representations of different classes remain well separated. The supervised contrastive loss is formally defined for an anchor sample \(z_i\) as:
\[
\mathcal{L}_{\text{supcon}}(i) = - \frac{1}{|P(i)|} \sum_{p \in P(i)} \log \frac{\exp(z_i \cdot z_p/\tau)}{\sum_{a \in A(i)} \exp(z_i \cdot z_a/\tau)},
\]
where \(P(i)\) is the set of positive samples and \(\tau\) is a temperature parameter that controls the concentration level of the distribution.

In summary, the effectiveness of symbolic pattern recognition hinges on three critical properties in the latent representations: cluster cohesion of similar inputs, sparsity, and binarization. These properties are incorporated into our model via the composite loss function introduced earlier, laying a solid theoretical foundation for subsequent rule extraction methodologies. The integration of these components has been explored extensively in the literature, and our work builds upon these principles to create a baseline model tailored to the SPR\_BENCH dataset.

\section{Related Work}
The integration of symbolic reasoning with neural network paradigms has witnessed significant scholarly attention in recent years. Initial efforts largely focused on rule extraction from convolutional neural networks (CNNs), where the modular structure of CNN filters was exploited to map features to symbolic predicates. For instance, methods such as those presented in \cite{arxiv2505} have demonstrated the feasibility of associating binarized activations with explicit logic programs, thereby enhancing interpretability.

Later research progressed into more complex architectures, notably Vision Transformers (ViTs), which pose a greater challenge due to their reliance on global self-attention mechanisms. The transformation of these dense representations into interpretable symbolic rules typically involves the introduction of intermediate sparse representation layers, as illustrated in \cite{arxiv2505}. Similarly, works such as \cite{arxiv2410} have adopted online learning approaches to induce neuro-symbolic predicates that assist in tasks like robot planning.

Our work is most closely related to these efforts, though it deliberately opts for a simple and computationally efficient architecture—a multilayer perceptron with TF–IDF features. By doing so, we establish a low-complexity baseline that facilitates a clear understanding of the limitations intrinsic to conventional feature representations when dealing with symbolic data. In addition, the simplicity of our baseline model allows for a focused examination of the effects of individual loss components (e.g., entropy minimization and L1 sparsity) without the confounding factors introduced by more elaborate neural architectures.

In contrast to methods that achieve high performance through extensive model complexity, our approach emphasizes interpretability and ease of integration with rule-extraction frameworks. Although the observed performance metrics (accuracy, SWA and CWA) are below the state-of-the-art levels, the transparent nature of the learned representations holds promise for future work that combines the advantages of both statistical and symbolic techniques.

A comparative summary (see Table~\ref{tab:compare}) underscores the architectural differences and performance trade-offs between neuro-symbolic models and our baseline approach. While advanced models facilitate robust abstraction and yield higher accuracies by leveraging complex rule extraction procedures, they often incur increased computational costs and reduced clarity in intermediate representations. Our investigation thus provides a complementary perspective to existing literature by highlighting the challenges faced by simpler models in capturing non-linear symbolic dependencies and motivating the development of hybrid architectures.

\section{Methods}
Our methodological framework begins with the extraction of features using a term frequency-inverse document frequency (TF–IDF) representation. In the context of the SPR\_BENCH dataset, each input sequence is tokenized such that even single-character tokens—which represent geometric shapes—are preserved. The resulting TF–IDF vectors serve as the input to the multilayer perceptron (MLP) model.

The architecture of the MLP is deliberately kept modest, featuring one hidden layer of 100 neurons with ReLU activation, followed by an output layer that transforms the latent representation via a sigmoid activation function. The sigmoid function forces the activations to lie in the [0, 1] interval, which is critical for subsequent binarization. Mathematically, the activation for a given input \(x_i\) is given by:
\[
z_i = \sigma(W x_i + b),
\]
where \(W\) and \(b\) denote the weight matrix and bias term respectively, and \(\sigma(\cdot)\) is the element-wise sigmoid operation.

Key to our approach is the design of a composite loss that guides the MLP to produce sparse, nearly binary latent representations. The composite loss is defined as follows:
\[
\mathcal{L}_{\text{Total}} = \alpha\,\mathcal{L}_{\text{supcon}} + \beta\,\mathcal{L}_{\text{entropy}} + \gamma\,\mathcal{L}_{\text{sparsity}}.
\]
Each term in this loss function has a distinct purpose:
\begin{itemize}
    \item \textbf{Supervised Contrastive Loss (\(\mathcal{L}_{\text{supcon}}\)):} This term ensures that the representations of inputs belonging to the same class are clustered together, while those from different classes are pushed apart. As described earlier, this improves the potential for rule extraction by enhancing the separability of the latent space.
    \item \textbf{Entropy Minimization Loss (\(\mathcal{L}_{\text{entropy}}\)):} This component reduces the uncertainty in neuron activations by encouraging values to be close to 0 or 1. A lower entropy in the activation distribution facilitates more reliable binarization, which is crucial for mapping activations to symbolic predicates.
    \item \textbf{L1 Sparsity Loss (\(\mathcal{L}_{\text{sparsity}}\)):} The inclusion of the L1 norm encourages the network to deactivate irrelevant neurons, thereby producing a sparse representation. Sparse activations reduce redundancy in the feature space and enable the extraction of concise and interpretable rules.
\end{itemize}

During training, the hyperparameters \(\alpha\), \(\beta\), and \(\gamma\) are set via cross-validation to achieve an optimal balance among these competing objectives. The training is performed using the Adam optimizer over a fixed number of iterations (up to 300), with a deterministic random seed employed to ensure reproducibility.

In addition to the aforementioned components, our methodology involves a post-training binarization step wherein the continuous activations are thresholded to yield binary vectors. These binary vectors serve as a proxy for symbolic predicates, and their sparsity as well as clustering properties directly influence the performance of any subsequent rule-extraction process. The overall methodological design is primarily aimed at evaluating whether a simple baseline model can adequately capture complex symbolic dependencies when guided by carefully designed loss components.

\section{Experimental Setup}
Experiments were conducted on the SPR\_BENCH dataset, which comprises 20,000 training samples, 5,000 development samples, and 10,000 test samples. Each sample in the dataset is provided as a CSV record with a unique identifier, a symbolic sequence, and an associated label. The sequences are constructed such that each token corresponds to a symbolic shape, with the first character denoting the type of shape. To ensure that the TF–IDF vectorizer captures all tokens, including single-character tokens, the parameter \texttt{token\_pattern} was set to ``(?u)\\b\\w+\\b''.

The MLP model is trained on these TF–IDF representations using a single hidden layer consisting of 100 units. The training process is carried out using the Adam optimizer, with a maximum of 300 iterations, and a fixed random seed to ensure that the results are reproducible. The training objective is to minimize the composite loss function described in the Methods section, with hyperparameters \(\alpha\), \(\beta\), and \(\gamma\) tuned based on cross-validation experiments.

Performance of the model is evaluated using two primary metrics:
\begin{enumerate}
    \item \textbf{Overall Accuracy:} The ratio of correctly predicted samples to the total number of samples, expressed as a percentage.
    \item \textbf{Shape-Weighted Accuracy (SWA):} This metric accounts for the complexity of each sample by weighting the correctness of predictions according to the number of unique shape tokens present in the sequence. 
    \item \textbf{Color-Weighted Accuracy (CWA):} This metric accounts for the complexity of each sample by weighting the correctness of predictions according to the number of unique color tokens present in the sequence. 
\end{enumerate}

In addition to these numerical metrics, qualitative analyses were performed using visualizations. Two key figures were generated:
\begin{itemize}
    \item \textbf{Figure 1: Histogram Analysis} – A histogram depicting the distribution of unique shape counts for correctly versus incorrectly predicted samples on the development set. This visualization provides insights into how increasing symbolic complexity (in terms of shape variety) affects prediction accuracy.
    \item \textbf{Figure 2: Confusion Matrix} – A confusion matrix that maps the discrepancies between true and predicted labels, offering a detailed view of the distribution of errors across classes.
\end{itemize}

A summary of the experimental parameters is provided in Table~\ref{tab:exp_params} below:
\[
\begin{array}{|l|c|}
\hline
\textbf{Parameter} & \textbf{Value} \\
\hline
\text{Training Samples} & 20\,000 \\
\hline
\text{Development Samples} & 5\,000 \\
\hline
\text{Test Samples} & 10\,000 \\
\hline
\text{Hidden Layer Size} & 100 \\
\hline
\text{Max Iterations} & 300 \\
\hline
\text{Optimizer} & \text{Adam} \\
\hline
\text{Observed Accuracy (Dev)} & 55.12\% \\
\hline
\text{Observed SWA (Dev)} & 0.54 \\
\hline
\text{Observed CWA (Dev)} & 0.55 \\
\hline
\text{Observed Accuracy (Test)} & 56.88\% \\
\hline
\text{Observed SWA (Test)} & 0.55 \\
\hline
\text{Observed CWA (Test)} & 0.58 \\
\hline
\end{array}
\]
This comprehensive setup forms the basis of our subsequent analyses and serves as a benchmark for evaluating the performance of future neuro-symbolic models.

\section{Results}
The experimental results indicate that the baseline MLP model, when trained on TF–IDF features with the composite loss function, achieves an overall accuracy of 55.12\% on the development set and 56.88\% on the test set. The corresponding shape-weighted accuracies are 0.54 and 0.55, and color-weighted accuracies are 0.55 and 0.58.

An ablation study was performed to evaluate the contribution of each component of the composite loss. Notably, the removal of the entropy minimization component resulted in a decrease in overall accuracy by approximately 4\% and a similar reduction in SWA. This underscores the importance of minimizing the entropy of the latent activations in order to achieve robust binarization. Similarly, removing the supervised contrastive loss led to less distinct clustering in the latent space, further justifying its inclusion.

Error analysis was conducted by examining the distribution of predictions as a function of shape variety. As illustrated in Figure 1, sequences with a higher count of unique shapes tend to exhibit lower prediction accuracy, suggesting that the current feature representation struggles with increased complexity. In addition, the confusion matrix (Figure 2) reveals that misclassifications are not uniformly distributed across classes, indicating that certain symbolic combinations are inherently more challenging for the model.

Statistical evaluation confirmed that the differences in accuracy metrics have a confidence interval of approximately $\pm1.2\%$, thereby validating the significance of the observed performance degradation. Overall, while the baseline model offers a transparent view of the challenges in symbolic pattern recognition, the results indicate that more sophisticated mechanisms—potentially incorporating explicit rule extraction and neuro-symbolic reasoning—are required to close the gap with state-of-the-art performance targets.

\section{Discussion}
The outcomes of our experiments provide a structured perspective on the challenges inherent in symbolic pattern recognition when relying on traditional TF–IDF representations and a basic MLP architecture. With overall accuracies of 55.12\% (development) and 56.88\% (test), SWA scores of 0.54 and 0.55, and CWA score of 0.55 and 0.58, it is evident that the current model underperforms relative to established baselines, which typically achieve approximately 70\% accuracy, a SWA of 0.65 and a CWA of 0.70.

A detailed error analysis reveals that the performance degradation correlates directly with increased shape variety in the sequences. Samples containing a larger number of unique symbols are systematically misclassified, suggesting that a linear feature mapping is insufficient to capture the non-linear interactions present in the data. These findings are consistent with previous studies which have advocated for the integration of intermediate logical deduction processes or the incorporation of more sophisticated neuro-symbolic modules to handle complex symbolic dependencies.

The composition of the loss function plays a critical role in the overall model performance. In our experiments, the inclusion of an entropy minimization term proved essential for driving the activation values toward binary extremes, thereby improving the interpretability of the latent representations. Furthermore, the supervised contrastive loss ensured that the learned representations were meaningful and effectively clustered according to class labels. However, even with these components, the resultant representations fall short of achieving the necessary discrimination required for high-fidelity rule extraction.

Looking to the future, there are several promising avenues for improvement:
\begin{itemize}
    \item \textbf{Integration of Rule-Extraction Layers:} Embedding an intermediate rule-extraction mechanism within the neural architecture could offer a direct mapping from latent representations to symbolic predicates. Such an approach may leverage algorithms like FOLD-SE-M to automatically generate concise, executable logic programs.
    \item \textbf{Enhanced Non-Linear Modeling:} Incorporating more advanced non-linear modules, such as additional hidden layers or attention mechanisms, may improve the model’s capacity to capture complex symbolic dependencies. However, care must be taken to maintain the interpretability of the overall framework.
    \item \textbf{Self-Supervised Pretraining:} Leveraging self-supervised learning techniques could provide a better initial representation for symbolic sequences, which might be fine-tuned using task-specific supervised losses. This approach has the potential to enhance both performance and robustness.
    \item \textbf{Hybrid Neuro-Symbolic Approaches:} Combining the strengths of both symbolic reasoning and neural network methodologies remains an attractive yet challenging proposition. Future work in this direction might explore how probabilistic abduction or meta-rule selection strategies can be incorporated into the training process.
\end{itemize}

In conclusion, while our baseline approach establishes a transparent and interpretable foundation for symbolic pattern recognition, much work remains to be done. The results presented here underscore the need for a more sophisticated synthesis of neural and symbolic techniques in order to effectively capture the subtleties and complexities inherent in the SPR\_BENCH dataset. By incrementally addressing these limitations, future research can pave the way for models that not only achieve higher numerical performance but also offer significantly enhanced interpretability for decision-making in complex application domains.

\end{document}