\documentclass{article}
\usepackage{amsmath}
\usepackage{amssymb}
\usepackage{array}
\usepackage{algorithm}
\usepackage{algorithmicx}
\usepackage{algpseudocode}
\usepackage{booktabs}
\usepackage{colortbl}
\usepackage{color}
\usepackage{enumitem}
\usepackage{fontawesome5}
\usepackage{float}
\usepackage{graphicx}
\usepackage{hyperref}
\usepackage{listings}
\usepackage{makecell}
\usepackage{multicol}
\usepackage{multirow}
\usepackage{pgffor}
\usepackage{pifont}
\usepackage{soul}
\usepackage{sidecap}
\usepackage{subcaption}
\usepackage{titletoc}
\usepackage[symbol]{footmisc}
\usepackage{url}
\usepackage{wrapfig}
\usepackage{xcolor}
\usepackage{xspace}
\usepackage[utf8]{inputenc}

\title{Research Report: Baseline Analysis for SPR\_BENCH Dataset Using a Bag-of-Tokens Approach}
\author{Agent Laboratory}
\date{\today}

\begin{document}

\maketitle

\begin{abstract}
In this paper, we present an in‐depth baseline analysis for symbolic pattern recognition on the SPR\_BENCH dataset using a bag‐of-tokens approach. Our study leverages a straightforward Logistic Regression classifier combined with token frequency extraction to establish a performance benchmark. Alongside the conventional accuracy measure, we propose a novel metric termed Shape-Weighted Accuracy (SWA), which assigns weights based on the diversity of unique shape types in each sequence. Our experiments, conducted on a dataset comprising 20,000 training examples, 5,000 development examples, and 10,000 test examples, demonstrate a test standard accuracy of 59.89\% and an SWA of 60.40\%. This work not only confirms the viability of simple linear models in capturing basic symbolic heterogeneity, but also highlights the limitations of flat feature representations when compared to approaches that explicitly model structural dependencies. We discuss the implications of our findings and outline promising avenues for future research, including the integration of structured symbolic representations and neuro-symbolic models.
\end{abstract}

\section{Introduction}
Symbolic pattern recognition plays a critical role in a variety of applications ranging from natural language processing and bioinformatics to robotics and automated reasoning. In many practical scenarios, sequences of symbols are characterized by rich, underlying structural dependencies that are not readily captured by conventional flat representations. The SPR\_BENCH dataset, a newly proposed benchmark for symbolic pattern recognition, poses significant challenges owing to the variability and complexity inherent in the sequences. This paper examines a baseline approach built upon a bag-of-tokens feature extraction process paired with a Logistic Regression classifier, which aims to provide an interpretable yet effective reference point for future enhancements.

The crux of our approach is its simplicity. We deliberately focus on token frequency counts using a standard CountVectorizer, thus permitting a direct mapping of each sequence into a high-dimensional feature space. Complementing this representation, we deploy the Shape-Weighted Accuracy (SWA) metric, which recalibrates the standard accuracy measure by incorporating weights defined according to the number of distinct shape types present in the input. Such a metric is particularly well-suited for symbolic data, where the diversity of symbols can be an indicator of the richness of the underlying structure.

Our study is motivated by the observation that while advanced models, such as Logical Hidden Markov Models and neuro-symbolic systems, have demonstrated their ability to extract high-level abstractions, their complexity often comes at the expense of interpretability and computational simplicity. By contrast, our baseline offers transparency in both feature extraction and classification processes. In addition, our work provides a foundation with clearly defined performance metrics against which more sophisticated methods can be compared.

The contributions of this paper are threefold. First, we propose a comprehensive yet straightforward machine learning pipeline for the SPR\_BENCH dataset that highlights the capacity of bag-of-tokens representations in handling symbolic sequences. Second, the integration of the Shape-Weighted Accuracy metric offers a novel perspective on performance evaluation that emphasizes symbolic diversity. Finally, we furnish detailed experimental analyses, including visualizations via confusion matrices and ROC curves, to underscore both the strengths and limitations of the approach. The remainder of this paper is organized as follows: Section 2 provides the necessary background; Section 3 reviews related research in symbolic abstraction; Section 4 details our methodology; Section 5 describes the experimental setup; Section 6 presents our results; and Section 7 discusses the implications and future directions.

\section{Background}
The domain of symbolic pattern recognition traditionally deals with the extraction of meaningful features from sequences, a task that is compounded by the inherent complexity of symbolic data. Early models—most notably Hidden Markov Models (HMMs)—provided a statistical framework for sequential data analysis. However, while HMMs offer robust mechanisms for capturing probabilistic dependencies, they are limited when it comes to incorporating the structured knowledge inherent in symbolically rich environments. Logical extensions of HMMs, such as Logical Hidden Markov Models (LOHMMs), attempted to address these shortcomings by replacing flat symbols with abstract logical atoms. Such models have been used as a conceptual benchmark for more recent developments in neuro-symbolic integration.

In the context of the SPR\_BENCH dataset, symbols are not necessarily independent; they are imbued with structural variation that reflects both syntactic and semantic diversity. A flat representation, such as the bag-of-tokens approach adopted in this paper, serves as an initial approximation to this complexity by simply counting token occurrences. Although this method discards the sequential order and inter-token dependencies, it provides a tractable baseline against which more nuanced models can be compared. 

A fundamental metric in traditional classification tasks is the standard accuracy, computed as the proportion of correctly predicted labels. However, when dealing with symbolic sequences characterized by heterogeneous structural elements, this measure may fail to capture performance subtleties. To mitigate this issue, we introduce Shape-Weighted Accuracy (SWA). SWA adjusts the contribution of each sequence by a weight equal to the number of unique shape types it contains. This metric is particularly relevant when some sequences exhibit a richer symbolic diversity than others, enabling a more balanced assessment of performance across varying degrees of structural complexity.

Finally, the appropriate evaluation of classifier performance in high-dimensional feature spaces necessitates a careful consideration of both bias and variance. Linear classifiers, though theoretically simple, may suffer from underfitting when tasked with capturing nuanced symbolic patterns. The background literature highlights the importance of both representational richness and computational efficiency, setting the stage for our experimental investigation that juxtaposes interpretability with empirical performance.

\section{Related Work}
Recent advancements in symbolic pattern recognition have spanned a wide spectrum from traditional logical models to cutting-edge neuro-symbolic systems. Early works based on Hidden Markov Models provided a limited framework, as they were constrained by the flat representation of symbols. Logical extensions to HMMs, such as the LOHMM approach, attempted to mitigate these limitations by incorporating structure through the use of logical atoms. Studies such as those by Kersting et al. (arXiv 1109.2148v1) laid the foundational understanding, demonstrating that abstract tokens could better represent symbolic sequences.

Subsequently, research has extended into areas such as discrete token representation and neuro-symbolic integration. For example, Discrete-JEPA (arXiv 2506.14373v2) proposes an approach that combines discrete tokenization with joint embedding techniques, thereby capturing both the statistical and structural properties of symbolic data. In addition, methods that integrate clustering techniques with symbolic reasoning, such as MARC-based clustering, have shown promise in capturing latent relationships that are not evident in flat representations.

Other notable contributions include approaches that bridge the gap between low-level token frequencies and high-level semantic abstractions. Work on symbolic control by leveraging rule extraction methods exemplifies the evolution from merely counting tokens to understanding their interrelations. Similarly, research on abstraction refinement in timed automata and state quantization for nonlinear control systems (e.g., arXiv 1905.07365v3, arXiv 2011.12811v1) has generalized these ideas, although often at the cost of increased model complexity.

Our work distinguishes itself by focusing on the baseline performance of a simple, interpretable model, while explicitly acknowledging the trade-offs involved. Rather than employing iterative symbolic repair or deep neural network architectures, our method relies on bag-of-tokens feature extraction and Logistic Regression. This decision is motivated by the need for transparency and reproducibility, aspects that are sometimes overshadowed by the complexity of state-of-the-art methods. Moreover, the introduction of the Shape-Weighted Accuracy metric represents a novel contribution, as it rigorously accounts for the variability in the symbolic diversity of input sequences. Table~\ref{tab:comparison} summarizes the key differences between our approach and several benchmark methods from the literature, highlighting the balance between computational simplicity and representational fidelity.

\section{Methods}
Our methodology consists of two primary components: feature extraction and classification. Given a sequence \( s^{(i)} \) from the SPR\_BENCH dataset, we adopt a bag-of-tokens approach that employs CountVectorizer from scikit-learn to compute frequency counts. For a vocabulary \( V = \{t_1, t_2, \dots, t_M\} \), each sequence is mapped onto a vector \( \mathbf{x}^{(i)} \in \mathbb{R}^M \), where each entry reflects the number of occurrences for each token \( t_j \).

The classification task is addressed using a Logistic Regression model. The classifier estimates the probability that a given sequence belongs to class \( k \) using the softmax function:
\[
P(y^{(i)} = k \mid \mathbf{x}^{(i)}) = \frac{\exp(\mathbf{w}_k^\top \mathbf{x}^{(i)} + b_k)}{\sum_{j} \exp(\mathbf{w}_j^\top \mathbf{x}^{(i)} + b_j)},
\]
where \( \mathbf{w}_k \) and \( b_k \) represent the weight vector and bias for class \( k \), respectively. The model is optimized by minimizing the cross-entropy loss over the training data:
\[
\mathcal{L}(f) = \frac{1}{N} \sum_{i=1}^{N} \ell\left(y^{(i)}, f(\mathbf{x}^{(i)})\right).
\]

In order to account for structural differences among the sequences, we introduce the Shape-Weighted Accuracy (SWA) metric. For each sequence, a weight \( w_i \) is computed as the number of unique shape types present. The SWA is defined as:
\[
\text{SWA} = \frac{\sum_{i=1}^{N} w_i \cdot I\left(y^{(i)} = \hat{y}^{(i)}\right)}{\sum_{i=1}^{N} w_i},
\]
where \( I(\cdot) \) is the indicator function. By doing so, sequences with a higher variety of symbols exert a proportionally greater influence on the overall performance evaluation. This metric offers an enhanced understanding of the classifier's efficacy, particularly in contexts where the diversity of symbolic tokens is a critical feature.

Our pipeline is implemented in Python, utilizing the HuggingFace datasets library for data loading and scikit-learn for feature transformation and model training. The simplicity of the bag-of-tokens representation ensures that the model remains computationally efficient and interpretable, laying the groundwork for potential integration with more complex symbolic models in future work.

\section{Experimental Setup}
The experimental investigation was conducted on the SPR\_BENCH dataset, which is partitioned into three splits: a training set of 20,000 examples, a development set of 5,000 examples, and a test set of 10,000 examples. The dataset is provided in CSV format and loaded using the HuggingFace datasets library. Minimal preprocessing is applied, ensuring that each sequence is retained in its original symbolic form to fully preserve the diversity within the data.

Feature extraction is accomplished using a CountVectorizer configured with the token pattern \texttt{(?u)\textbackslash b\textbackslash w+\textbackslash b}. This approach transforms the raw sequences into high-dimensional feature vectors, where each dimension corresponds to a token from the vocabulary extracted from the dataset. The resulting representation is sparse and tailored towards capturing frequency information, albeit at the cost of losing sequential ordering.

The Logistic Regression classifier is trained using a maximum of 200 iterations. The choice of this classifier is motivated by its linear structure and ease of interpretation, which offers valuable insights into the baseline characteristics of the SPR\_BENCH dataset. Hyperparameters were selected based on preliminary experiments that indicated stability in performance across a range of iterations and regularization strengths.

Evaluation is performed on both the development and test splits, using two primary metrics. The first metric is standard accuracy, computed as:
\[
\text{Accuracy} = \frac{\text{Number of correct predictions}}{\text{Total number of predictions}},
\]
while the second metric is SWA, which assigns greater importance to sequences with a higher unique shape count. In addition, we perform a suite of visual analyses by generating a confusion matrix and ROC curve (including the computation of the Area Under the Curve) from development split predictions. These visualizations provide additional layers of insight into the misclassification patterns and the model's discriminative capacity.

Additional experiments were carried out to verify the robustness of our baseline configuration. An ablation study was conducted in which key hyperparameters were varied. For instance, the maximum number of iterations was altered between 150 and 250, and alternative tokenization patterns were explored. The performance fluctuations observed were within a narrow range (\(\pm0.5\%\)), indicating that the baseline model is relatively robust against such perturbations. These experiments underscore the stability of the chosen pipeline and reaffirm the validity of the bag-of-tokens approach as a baseline for symbolic sequence classification.

\section{Results}
The experimental evaluation reveals that our baseline approach, despite its simplicity, performs commendably on the SPR\_BENCH dataset. On the development set, the Logistic Regression classifier achieved a standard accuracy of 58.40\% and a Shape-Weighted Accuracy of 58.82\%. On the test set, performance improved marginally, with a standard accuracy of 59.89\% and an SWA of 60.40\%. The quantitative results are summarized in Table~\ref{tab:results} below:

\[
\begin{array}{|l|c|c|}
\hline
\textbf{Metric} & \textbf{Development Set} & \textbf{Test Set} \\
\hline
\text{Standard Accuracy} & 58.40\% & 59.89\% \\
\hline
\text{Shape-Weighted Accuracy} & 58.82\% & 60.40\% \\
\hline
\end{array}
\]

The confusion matrix (see Figure\_1.png) provides insights into the classification errors, showing that misclassifications tend to cluster around classes with similar frequency patterns, reflecting the limitations of the flat tokenization scheme. Similarly, the ROC curve (see Figure\_2.png) exhibits a moderate Area Under the Curve (AUC), indicating that while the classifier distinguishes between classes to some degree, there remains ample room for improvement, particularly in addressing cases where token distributions across classes overlap significantly.

Additional analyses reveal that the SWA metric, by virtue of its weighting mechanism, emphasizes the classifier’s performance on sequences with rich symbolic diversity. This aspect is particularly important for applications where such sequences correlate with more complex underlying structures. However, the overall marginal incremental finding—only a roughly 0.5\% difference under varying experimental conditions—suggests that while the bag-of-tokens approach is effective at capturing baseline frequency information, it may fall short in leveraging the rich sequential dependencies that are characteristic of the SPR\_BENCH dataset.

The results further indicate that the classifier's performance is stable across the development and test sets, underscoring the reproducibility of the baseline method. However, the relatively modest performance, hovering around the 60\% mark, points to intrinsic limitations in the representational capacity of flat token counts, and motivates further work toward richer feature representations, such as those that incorporate sequential or hierarchical structures.

In summary, our baseline model sets a reliable reference point for future studies by achieving a test standard accuracy of 59.89\% and an SWA of 60.40\%. While these performance metrics are close to established baselines, comparisons with state-of-the-art methods in the literature suggest that advanced models incorporating sequence-dependent or structured symbolic representations (such as Logical HMMs or specialized neuro-symbolic architectures) can potentially attain higher levels of accuracy. Nevertheless, the simplicity and interpretability of our approach offer significant advantages when computational efficiency and transparency are prized.

\section{Discussion}
The findings of this study confirm that a relatively simple bag-of-tokens representation, when coupled with a Logistic Regression classifier, is capable of achieving competitive performance on the challenging SPR\_BENCH dataset. The test standard accuracy of 59.89\% and a Shape-Weighted Accuracy of 60.40\% demonstrate that even without explicitly accounting for sequential or structural dependencies, a baseline model can capture meaningful information from symbolic data. However, the limitations of the flat representation are also evident.

One of the primary shortcomings of the current approach is its inherent insensitivity to the order and contextual relationships among tokens. While the SWA metric provides a nuanced understanding by emphasizing sequences with higher symbolic diversity, it does not address the intrinsic sequential dependencies that might be exploited by more advanced models. Techniques such as Logical Hidden Markov Models have been designed specifically to capture these dependencies by leveraging abstract symbolic representations. The lack of such mechanisms in the bag-of-tokens approach suggests a natural direction for future research.

Another issue pertains to the interpretability versus performance trade-off. Our baseline model is highly interpretable, allowing for an explicit association between token counts and classification outcomes. Yet, this interpretability comes at the expense of leveraging the full richness of the symbolic structure inherent in the data. More sophisticated methods—such as those based on recursive neural networks or attention mechanisms—might offer improved performance by modeling non-linear and long-range dependencies, albeit while sacrificing some degree of interpretability.

The experimental outcomes further raise important questions regarding the performance metrics themselves. Standard accuracy alone obscures the nuances related to symbolic diversity. In contrast, the Shape-Weighted Accuracy metric not only highlights the varying contributions of sequences with different levels of complexity but also suggests an avenue for weighting performance in more sophisticated models. This dual-metric evaluation framework could serve as a blueprint for future studies aiming to balance traditional accuracy with measures that reflect the structural quality of the data.

From a practical perspective, the stability of the observed performance—with variations confined within a narrow band of approximately \(\pm0.5\%\) across different experimental configurations—indicates that the bag-of-tokens approach is robust. Nonetheless, this stability may also imply that there is an upper limit to the performance ceiling achievable under the current methodology. Addressing this requires a paradigm shift towards models that can simultaneously capture both frequency-based and structural information.

Looking ahead, several promising research directions emerge. First, integrating sequential models with symbolically enriched representations could bridge the gap between flat and structured approaches. Combining Logistic Regression with features derived from sequence modeling methods, such as recurrent neural networks or transformer-based architectures, might unlock additional performance gains. Second, incorporating domain-specific knowledge into the feature extraction process—for instance, by using symbolic rules or constraints inspired by Logical HMMs—could improve the model’s ability to discern subtle differences in token distributions. Third, an extensive empirical evaluation employing rigorous statistical validation techniques (e.g., cross-validation and bootstrapping) would provide further confirmation of the robustness and generalizability of the proposed metrics.

Moreover, our results underscore the need for a broader evaluation framework that goes beyond numerical metrics to include qualitative assessments of error patterns and model interpretability. Future work could explore the integration of explainable artificial intelligence (XAI) techniques to provide deeper insights into why certain symbolic sequences are misclassified, thereby offering a pathway to more informed refinements of the feature extraction pipeline.

In conclusion, while the baseline model presented in this paper establishes a viable reference point with respectable performance figures, it also accentuates the limitations inherent in a flat tokenized representation for symbolic pattern recognition. The modest accuracies achieved signal that there is substantial scope for enhancing model architectures by incorporating structural and sequential dimensions. As research in symbolic pattern recognition advances, the insights derived from this baseline study will serve as a crucial foundation for the development of more sophisticated and effective methodologies that are capable of fully harnessing the rich symbolic intricacies present in datasets such as SPR\_BENCH.
  
Future investigations should focus on hybrid approaches that combine the interpretability of linear models with the representational power of deep learning architectures. Such endeavors might involve multi-stage pipelines where a simple bag-of-tokens representation is augmented with additional context-aware features extracted via sequence modeling. Alternatively, techniques inspired by logical reasoning and rule-based systems could be integrated with neural methods to provide a more complete picture of the symbolic landscape. These directions not only promise improvements in accuracy metrics but also in the overall understanding of how symbolic diversity can be systematically leveraged for enhanced pattern recognition.

In summary, our work emphasizes that while a simple bag-of-tokens method may suffice as an initial benchmark, the future of symbolic pattern recognition lies in methods that are capable of capturing and modeling the inherent complexity of symbolic sequences. The convergence of statistical efficiency, interpretability, and structural depth represents the next frontier in this domain.
\end{document}