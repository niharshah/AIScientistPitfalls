\documentclass{article}

% This is the ICBINB Workshop style, do not remove or alter.
% Packages and setup
\usepackage[margin=1in]{geometry}
\usepackage{graphicx}
\graphicspath{{figures/}}
\usepackage{amsmath,amssymb}
\usepackage{booktabs}
\usepackage{hyperref}

% Place references in filecontents
\begin{filecontents}{references.bib}
@misc{placeholder2025,
  title        = {Placeholder Reference},
  author       = {Author, A. N.},
  year         = {2025},
  note         = {No references provided in logs}
}
\end{filecontents}

\begin{document}

\title{Pitfalls in Color-Shape Classification}
\author{Anonymous Submission}
\date{}
\maketitle

\begin{abstract}
We investigate challenges in color-shape classification, revealing negative and inconclusive results that illustrate difficult real-world pitfalls. In practical settings, these issues hamper standard architectures and motivate better evaluation strategies.
\end{abstract}

\section{Introduction}
Color-shape classification can face persistent difficulties even when recent deep learning methods claim robust features. Our findings highlight cases where color cues fail to generalize. Although shape extraction may be consistent, color classification remains inconsistent.

\section{Related Work}
Prior studies have examined color invariance and feature disentanglement \cite{placeholder2025}. However, our focus on real-world color confusion extends these discussions by stressing problematic cases observed in simple classification tasks.

\section{Method Discussion}
We implement a baseline CNN and a transformer-based model. Both architectures are trained on synthetic color-shape data with controlled variations. We intentionally design color distributions that mimic real-world complexities.

\section{Experiments}
We measure accuracy on both shape and color labels in training and validation. Our baseline converges on shape features but often misclassifies color. The transformer model partially improves color recognition yet remains prone to confusion, especially for closely related hues.

\subsection*{Reflection on Figures and Layout}
Our main text includes two composite figures comparing the baseline and transformer, detailing loss curves, scatter plots, and confusion matrices. Each figure exemplifies how color accuracy remains problematic despite shape features converging well. Ablation tests examining image clustering, positional encoding, and pooling strategies are consolidated into a single appendix figure, offering deeper analysis without cluttering the main text.

\section{Conclusion}
Color-shape classification still suffers from misinterpretations of color cues. Our negative and inconclusive results underscore the importance of data distribution and feature analysis. Future work may explore specialized color embeddings or more sophisticated data augmentation to mitigate these pitfalls.

\bibliographystyle{plain}
\bibliography{references}

\appendix
\section{Additional Ablation Figures}
We consolidate experimental ablations (e.g., removing glyph clustering or positional encodings) into one consolidated figure here, indicating moderate gains or unstable training behaviors.

\end{document}