\documentclass[11pt]{article}
\usepackage[margin=1in]{geometry}
\usepackage{amsmath,amssymb}
\usepackage{graphicx}
\graphicspath{{figures/}}

%-----------------------
% References File
%-----------------------
\begin{filecontents}{references.bib}
@misc{smith2020example,
  title={An Example Paper on Negative or Inconclusive Results},
  author={Smith, John},
  year={2020},
  howpublished={arXiv preprint arXiv:2001.00001}
}
\end{filecontents}

\begin{document}

\title{Reflections on Figure Placement \& Page Limit Optimization}
\maketitle

\begin{abstract}
This document discusses how best to place figures in the main text and the appendix, focusing on minimizing redundancy while retaining important experimental insights. We also outline minor text updates that help conserve page space while still thoroughly covering essential findings.
\end{abstract}

\section{Introduction}
Figure usage must be carefully planned to highlight key results and support our central claims. In practice, many exploratory plots offer partial or duplicative insights, so identifying the minimum number of figures for the main text is vital. We aim to retain figures directly relevant to final accuracy, confusion patterns, or other crucial performance metrics. Less critical plots, such as additional ablations, are shifted to the appendix unless they provide unique information.

\section{Reflections on Figures}
\textbf{Figures in the Main Text.} Figure\,1 (with subfigures \texttt{baseline\_bwa\_curves\_part1.png} and \texttt{baseline\_test\_bwa\_comparison.png}) remains important for illustrating training/validation performance across epochs and the final test score. We will keep it consolidated to show how different epoch budgets affect both convergence and final results. Figure\,2 (\texttt{baseline\_confusion\_matrix\_max\_20.png}) is also essential, as it captures frequent misclassifications across categories.

\textbf{Figures Moved to the Appendix.} Nine extra plots (\texttt{attronly\_loss\_bwa.png}, \texttt{attronly\_test\_metrics.png}, \texttt{baseline\_bwa\_curves\_part2.png}, \texttt{research\_test\_metrics.png}, \texttt{single\_layer\_bwa.png}, \texttt{single\_layer\_cwa\_swa.png}, \texttt{single\_layer\_label\_distribution.png}, \texttt{single\_layer\_loss.png}, \texttt{unidirectional\_metrics.png}) show partial or duplicate information. They are moved to the appendix as supplementary figures. If any plot offers no new insight, it is removed to conserve space.

\textbf{Potential Figure Combinations.} Several related ablations can be merged into composite subfigure panels in the appendix, reducing clutter. For instance, single-layer vs.\ multi-layer comparisons can be grouped to emphasize how architecture depth influences accuracy without crowding the main text.

\textbf{Further Text Adjustments.} The main text gains concise commentary on Figure\,1 regarding the observed training plateau and practical reasons for choosing certain epoch ranges. The discussion around Figure\,2 highlights which categories (e.g., shape vs.\ color) are confused most often. In the appendix, supplementary plots are briefly described and referenced only when they extend the analysis of sub-configurations or ablations.

\section{Conclusion}
We preserve two essential figures in the main text: (1) training/validation curves plus final test comparison and (2) confusion matrix analysis. All remaining figures are relegated to the appendix unless they offer unique insights into unforeseen pitfalls. This strategy maximizes clarity in the main paper while keeping within the page limit. Future adaptations may focus on more compact, composite figure layouts to further streamline presentation.

\bibliographystyle{plain}
\bibliography{references}

\end{document}