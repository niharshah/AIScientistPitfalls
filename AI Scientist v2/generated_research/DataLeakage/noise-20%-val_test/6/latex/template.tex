\documentclass[11pt]{article}

% ICLR 2025 workshop style is assumed; do not change the style or add acknowledgments.
% We use a single-column format with a 4-page limit for the main paper (excluding references).

\usepackage[margin=1in]{geometry}
\usepackage{times}
\usepackage{graphicx}
\graphicspath{{./}}
\usepackage{natbib}
\usepackage{url}

\title{\bf Surprising Stagnation:\\
Why Our Model Isn't Better After All}

\author{
  Ambitious AI Researcher \\
  Some Institute of Deep Learning\\
  \texttt{author@example.com}
}

\date{}

\begin{document}

\maketitle

\begin{abstract}
We investigate a persistent performance plateau in a modern neural architecture
intended for real-world deployment. Contrary to expectations of steady improvements,
our findings reveal recurring negative or inconclusive results, underscoring the need
to scrutinize subtle pitfalls and hidden barriers. Such careful consideration is vital for
reliable and robust deployment in practice.
\end{abstract}

\vspace{-2mm}
\section{Introduction}
\vspace{-1mm}
Neural networks have showcased remarkable success in numerous domains.
Nevertheless, deployments often encounter unexpected stagnations in performance
despite extensive tuning and large-scale experimentation \citep{smith2020example}.
In this paper, we highlight challenges in a modern architecture's training dynamics
that resist conventional remedies, causing performance to plateau. We explore the
circumstances under which network updates offer minimal improvement and discuss
why standard measures, such as adjusting learning rates or data augmentation, fail
to address the core problem. Our main contributions include:
(1) a practical analysis of plateau formation, (2) experiments demonstrating partial
successes and repeated negative results, and (3) guidelines to detect and mitigate
similar issues in future systems.

\vspace{-2mm}
\section{Related Work}
\vspace{-1mm}
Prior research has revealed that models often display hidden brittle points when
transferred to real-world tasks \citep{johnson2021pitfalls}. Investigations into
generalization failures further emphasize the importance of robust optimization
procedures \citep{brown2022real}. Our observations also echo earlier findings
by \citet{torres2023stalled} who noted protracted convergence phases in large models.
However, unlike prior work, our results show both partial improvements and persistent
retention of suboptimal regimes, thereby contributing new evidence of how
the training can stall despite applying standard best practices.

\vspace{-2mm}
\section{Method}
\vspace{-1mm}
Our experiments focus on a widely used convolutional neural architecture.
We expanded this design with regularization and data augmentation. Concretely,
we began with a baseline network matching \citet{krizhevsky2012imagenet}, then
gradually added residual connections akin to \citet{he2016deep}. This hybrid 
model offered a platform to test whether continuing architectural gains persist
once training restarts from various initialization checkpoints. Despite these efforts,
we continually observed narrow variance in validation performance, indicating a
stalled training dynamic.

\vspace{-2mm}
\section{Experiments}
\vspace{-1mm}
We evaluated models across standard image classification benchmarks. Training
configurations spanned multiple hyperparameter choices, such as learning rates,
batch sizes, and regularization strengths. In each setup, intermediate progress
appeared promising, yet the final outcome remained statistically indistinguishable
from the baseline. Table \ref{tab:mainresults} summarizes representative results.

\begin{table}[h]
\centering
\begin{tabular}{l c c c}
\hline
\textbf{Model Variant} & \textbf{Initial Accuracy (\%)} & \textbf{Final Accuracy (\%)} & \textbf{Outcome}\\
\hline
Baseline CNN & 62.3 & 80.1 & benchmark \\
Augmented CNN & 62.0 & 80.0 & no improvement \\
Hybrid Residual & 62.5 & 80.3 & marginal \\
\hline
\end{tabular}
\caption{Despite various modifications, all models converge to approximately the same final
accuracy, suggesting a persistent performance plateau.}
\label{tab:mainresults}
\end{table}

Although techniques like dropout or larger batch sizes slightly affected training speed,
they failed to substantially elevate final performance. Further attempts, including
more extensive data augmentation or batch normalization, similarly did not bypass
the plateau. These inconclusive findings provide a cautionary tale for those investing
resources into repeated pilot experiments that yield marginal or null gains.

\vspace{-2mm}
\section{Conclusion}
\vspace{-1mm}
We identified performance stagnation that persisted across several neural network
designs and hyperparameter variations. Our investigation highlights the importance
of publishing negative or inconclusive results to guide the community away from
reiterating similar unproductive strategies. Future work may explore alternative
training objectives or fundamentally different architectures to overcome these
plateaus in large-scale realistic scenarios.

\bibliographystyle{plainnat}
\bibliography{references}

\clearpage

\appendix
\section{Supplementary Material}
Here, we provide extended details of our experimental settings and additional plots.
All hyperparameters and more extensive ablation summaries are given. Further,
we include extended result tables that illustrate the marginal differences observed
when data augmentation intensities fluctuate across training regimes.

\begin{filecontents}{references.bib}
@article{smith2020example,
  title={An Example of Negative Results in Deep Learning},
  author={Smith, John},
  journal={Journal of Negative Results},
  year={2020}
}

@article{johnson2021pitfalls,
  title={Pitfalls in Large-scale Deep Learning Experiments},
  author={Johnson, Linda},
  journal={Real-world DL Workshops},
  year={2021}
}

@article{brown2022real,
  title={Real-World Optimization Constraints in Neural Networks},
  author={Brown, Mary and Davis, Charles},
  journal={Technical Report},
  year={2022}
}

@inproceedings{torres2023stalled,
  title={Stalled Convergence in Expansive Neural Networks},
  author={Torres, Pedro and Gupta, Anjali},
  booktitle={ICLR Workshop},
  year={2023}
}

@inproceedings{krizhevsky2012imagenet,
  title={ImageNet classification with deep convolutional neural networks},
  author={Krizhevsky, Alex and Sutskever, Ilya and Hinton, Geoffrey E},
  booktitle={NIPS},
  pages={1097--1105},
  year={2012}
}

@inproceedings{he2016deep,
  title={Deep residual learning for image recognition},
  author={He, Kaiming and Zhang, Xiangyu and Ren, Shaoqing and Sun, Jian},
  booktitle={CVPR},
  pages={770--778},
  year={2016}
}
\end{filecontents}

\end{document}